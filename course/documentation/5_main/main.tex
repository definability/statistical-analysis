\chapter{Прогнозування часових рядів}

На вході дано часовий ряд $Y$ з шумом $E$
\begin{equation*}
  Y = Z + E,
\end{equation*}
де
\begin{equation*}
  cov(E_i, E_j) =
    \begin{cases}
      \sigma_e^2,& i = j, \\
      0,&          i \neq j.
    \end{cases}
\end{equation*}
Також наявні аномальні точки,
які заважають процесу зглажування та прогнозування.

\section{Видалення аномалій}

Для початку побудуємо зглажування на основі експоненційно зваженого
ковзкого середнього з $\alpha = \frac{1}{3}$,
при якому довжина віконця дорівнює $6$.
\begin{center}
  \adjustimage{max size={0.9\linewidth}{0.9\paperheight}}{Coursework_files/Coursework_12_0.png}
\end{center}

Одразу видно, де знаходяться аномальні точки,
проте треба застосувати метод,
який ґрунтується на здоровому глузді
та математичній статистиці.
Оскільки аномальні точки заважають будувати лінію тренду,
потрібно порівняти якість фільтрації при видаленні різних точок даного ряду.
Введемо $Y_{fixed}\left( t \right)$ як ряд, в якому ``виправлено'' точку $t$.
Це означає, що дані в цій точці мають значення, яке не є аномальним.
Під якістю фільтрації розуміємо середньоквадратичне відхилення
між лінією тренду та самим рядом.
Порівняємо якість зглажування для $Y$
з якістю зглажування його виправленого аналогу.
Логічно, що можна порівняти їх співвідношення з певним граничним значенням
\begin{equation*}
  V\left( t \right)
  = \frac{Var\left( Y - Y^{smoothed} \right)}
         {Var\left( Y_{fixed}\left(t\right)
                    - Y_{fixed}^{smoothed}\left(t\right) \right)}
  > V_{critical}
\end{equation*}
Отримали вираз, що відомий як $F$-тест:
маємо вибіркові дисперсії двох вибірок,
що за умовою мають нормальний закон розподілення.
Аномальні точки будемо знаходити одну за одною ---
знаходити найгіршу та виправляти її,
якщо вона дійсно аномальна
\begin{equation*}
  \max\limits_{t}{F_{F, T, T-1}\left( V\left( t \right) \right)}
    > F_{F, T, T-1}^{critical}
  \Longrightarrow
  t_{anomaly} = \max\limits_{t}{F_{F, T, T-1}\left( V\left( t \right) \right)}.
\end{equation*}
Поглянемо на те, як змінюється вибіркова дисперсія помилки
при відкиданні кожної точки.
\begin{center}
\adjustimage{max size={0.9\linewidth}{0.9\paperheight}}{Coursework_files/Coursework_15_0.png}
\end{center}

З рисунку видно піки саме в тих місцях,
де передбачалась наявнисть аномальних точок.
Проте зараз в нас є міра їх ``аномальності''
у вигляді ваги хвостів розподілу Фішера.
\begin{center}
\adjustimage{max size={0.9\linewidth}{0.9\paperheight}}{Coursework_files/Coursework_17_0.png}
\end{center}

Відкидаємо одну за одною точки, при яких дисперсія значно змінюється,
а саме такі, що з ймовірністю $0.9$ це дисперсія іншої виборки.

\begin{center}
\adjustimage{max size={0.9\linewidth}{0.9\paperheight}}{Coursework_files/Coursework_17_0.png}
\end{center}

\begin{center}
\adjustimage{max size={0.9\linewidth}{0.9\paperheight}}{Coursework_files/Coursework_18_0.png}
\end{center}

\begin{center}
\adjustimage{max size={0.9\linewidth}{0.9\paperheight}}{Coursework_files/Coursework_18_1.png}
\end{center}

\begin{center}
\adjustimage{max size={0.9\linewidth}{0.9\paperheight}}{Coursework_files/Coursework_18_2.png}
\end{center}

Аномальні дані знаходилися в точках $2$, $12$ і $4$.
Їх було виправлено за принципом найближчих сусідів ---
середнє арифметичне найближчих точок.
Маємо графік з даними без аномалій та зглажуванням,
яке тепер значно краще описує ряд

\begin{center}
\adjustimage{max size={0.9\linewidth}{0.9\paperheight}}{Coursework_files/Coursework_21_0.png}
\end{center}

\section{Вибір правильного зглажуючого віконця}

Для подальших обчислень потрібно обрати коректні параметри зглажування,
а саме параметр $\alpha$ для експоненційно зглаженого ковзкого середнього.
Зазвичай це робиться в залежності від природи даних ---
їх циклічності або характерного часу зміни.
Цієї інформації немає, проте ми знаємо, що помилки мають гаусовий розподіл.
Скористуємося відомим методом
для визначення ``нормальності'' розподілу виборки ---
методом Д'Ауґустіно Ральфа \cite{dago1990}.
Цей тест на виході дає відстань розподілу даної виборки
до класу нормально розподілених виборок.
Ми хочемо обрати таке зглажування,
щоб розподіл помилок був якомого більш гаусовим.
Потрібно обрати таке $\alpha$,
для якого відстань між розподілом помилок та нормальним розподілом
була наймешною.
Ця умова винонується для віконця шириною $7$,
що відповідає $\alpha = \frac{1}{4}$

\begin{center}
\adjustimage{max size={0.9\linewidth}{0.9\paperheight}}{Coursework_files/Coursework_24_0.png}
\end{center}
